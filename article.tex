\documentclass[fleqn,10pt]{SelfArx} % Document font size and equations flushed left

\setlength{\columnsep}{0.55cm} % Distance between the two columns of text
\setlength{\fboxrule}{0.75pt} % Width of the border around the abstract

\definecolor{color1}{RGB}{0,0,0} % Color of the article title and sections
\definecolor{color2}{RGB}{0,20,20} % Color of the boxes behind the abstract and headings

\newlength{\tocsep} 
\setlength\tocsep{1.5pc} % Sets the indentation of the sections in the table of contents
\setcounter{tocdepth}{3} % Show only three levels in the table of contents section: sections, subsections and subsubsections

\usepackage{float}
\usepackage{mdwlist}
\usepackage{caption}
\usepackage{verbatim}
\usepackage[T1]{fontenc}
\usepackage[utf8]{inputenc}

% Consider inserting:
%\usepackage{kpfonts}

\JournalInfo{\today} % Journal information
\Archive{First~Draft} % Additional notes (e.g. copyright, DOI, review/research article)
\PaperTitle{Emulated~GPGPU~Kernels\\A~Study~into~Performance} % Article title
\Authors{Eric~Nilsson} % Authors
\affiliation{EricNNilsson@gmail.com}
%\affiliation{*\textbf{Corresponding author}: john@smith.com} % Corresponding author
\Keywords{GPGPU, GPU, Performance, Emulation, Simulation} % Keywords - if you don't want any simply remove all the text between the curly brackets
\newcommand{\keywordname}{Keywords} % Defines the keywords heading name

% ABSTRACT
\Abstract{
\ldots \\
As such, the hypothesis suggested in this proposal may be summarized as follows:
\quote{\textit{The performance of an emulated GPGPU-kernel on a CPU may have it’s performance substantially improved in exchange for some computational precision.}}
% Background	(what)
%	Vilket sammanhang utgör studiens bakgrund?
% Challenge		(why)
%	Vilka utmaningar omfattar studien?
% Approach		(how)
%	Vilken ansats tillämpas i studien?
% Results 		(what)
%	Vilka resultat gav studien?
}

\begin{document}
\flushbottom % Makes all text pages the same height
\maketitle % Print the title and abstract box

\noindent
\textit{"Inspirational quote."}
\begin{center}
- Someone smart.
\end{center}

\tableofcontents % Print the contents section
\thispagestyle{empty} % Removes page numbering from the first page

% INTRODUCTION
\section*{Introduction} % The \section*{} command stops section numbering
\addcontentsline{toc}{section}{\hspace*{-\tocsep}Introduction} % Adds this section to the table of contents with negative horizontal space equal to the indent for the numbered sections
\label{sec:introduction}
% Background and Challenge
%	Vilket sammanhang utgör studiens bakgrund?
%	Vilka utmaningar omfattar studien?
When developing \textit{GPGPU}-kernels relatively small changes may induce large deviations in performance, due to massively parallelized instruction sets and architectural differences in-
between on-chip hardware. Therefore, it may be desirable for the developer to be able to view the data he or she is modifying on the graphics card - a possibility often limited in terms of \textit{GPGPU}-
technologies.\\
\\
% Approach and Result
%	Vilken ansats tillämpas i studien?
%	Vilka resultat gav studien?
The increased utilization of \textit{GPGPU} has brought forth the need of more extensive debugging-possibilities involving access of data that may be hard to retrieve from graphics cards - possibly due to architectural differences in-between chip manufacturers. A preferred solution to this problem has been to emulate such \textit{GPU}-kernels on the \textit{CPU} - in exchange for a substantial performance-loss. This study comprises an investigation into the performance of software emulation of hardware accelerated \textit{GPU}-kernels, by the means of analyzing several software rasterizers.\\
This study concerns an investigation into the DirectCompute-framework on the Windows-platform, analyzing the performance of a \textit{GPGPU}-kernel on-chip, using the DirectX standard software rasterizer, and utilizing the DirectX 11.1-addition\footnote{Bundled with Windows 8.} Microsoft \textit{WARP}-technology (Windows Advanced Rasterization Platform) - which promises high-speed software emulation in exchange for some precision. \\
A logical next step of such an evaluation would be to investigate any influence Microsoft WARP may have on computational precision. Thus, it is important that the result such a kernel would produce is measurable and deterministic. \\
\\
This study concludes that the performance loss of emulated \textit{GPGPU}-kernel may be considered negligible in exchange for possible extended possibilities of data extraction and increased granularity. As such, Microsoft \textit{WARP}-technology may be considered feasible for use in industry rasterization\footnote{Such a use-case might be if hardware is unavailable, not suffiecient or busy - as suggested by Microsoft. Link to Microsofts page on WARP here!}.\\
\\
% Outline and Conclusion
%	Vilket upplägg har resterande delar av matrialet?
As such, the study concerns the fields of simulation, emulation and \textit{GPU}-technologies - respectively, with the purpose of facilitating debugging and profiling of \textit{GPU}-kernels, whilst maintaining acceptable performance.

% CONTRIBUTION
\section{Contribution}
\label{sec:contribution}

% CONTRIBUTION - METHOD
\subsection{Method}
\label{sec:contribution:method}
% Introduction
%	Vilka principer, modeller, metoder och teknologier omfattar studiens design?
The experiment performed includes a DirectCompute \textit{GPGPU}-kernel consisting of a large matrix-multiplication which is calculated, in it’s entirety, in aforementioned kernel. The motivation behind such a task is that the result is easily verified as correct, and the precision of such an operation can be tested with different data-types - with regard to the effect Microsoft~\textit{WARP} may have on computational precision.\\
The test-cases investigated may be summarized as follows:
% Describe these modes and what they entail:
\begin{description*}
\item[Hardware-Acceleration:] \hfill \\
	The execution of a DirectCompute-kernel on a graphics card. Thus, this case will act as a reference for the remaining experiments.\\
	Expected high performance.
\item[Windows~Advanced~Rasterization~Platform~(\textit{WARP}):] \hfill \\
	The emulated execution of a DirectCompute-kernel using a special software rasterizer devised by Microsoft in their latest revision of the DirectX-framework.\\	
	Expectations unclear, but expected to perform better than standard software rasterization.
\item[Software~Rasterization~(DirectX~Reference~Device):] \hfill \\
	The emulated execution of a DirectCompute-kernel using a software rasterizer.\\
	Expected poor performance.
\end{description*}
In addition to these scenarios, the performance of several kernels with varying lavel of optimization have been examined. These entail the following:
\begin{enumerate*}
	\item Basic un-optimized kernel.
	\item \ldots
\end{enumerate*}
% Un-optimized basic kernel.
% Tiled kernel basic kernel.
% Tiled kernel utilizing shared memory.
% Tiled kernel utilizing shared memory with memory coalescing.
Furthermore, all test-cases have been performed with both integer- and floating-point precision\footnote{Not yet!}. As such, the experiment has been devised of the following approximate steps for each scenario:
\begin{enumerate*}
	\item Randomize two matrices \textbf{A} \& \textbf{B} using the specified data-type. These matrices will be the subjects of the experiment.
	\item Establish the product-matrix \textbf{Ref} of the two matrices. The resulting matrix will be used as a reference matrix to verify the final result.
	\item Start a synchronized high-precision timer.
	\item Dispatch the DirectCompute-kernel calculating the product matrix \textbf{A}\textbf{B}=\textbf{C}
	\item Stop the timer once the DirectCompute-kernel has finished.
	\item Assure that the resulting matrix is correct by comparing \textbf{C} to the previously established reference-matrix \textbf{Ref}.
\end{enumerate*}

% CONTRIBUTION - TECHNOLOGIES
\subsection{Technologies}
\label{sec:contribution:technologies}
The experiment was subdivided into two major components, both of which use Microsoft~Visual~Studio~2012 for compilation. These are presented below.

\subsubsection*{matrixgen}
The first, \textit{matrixgen},  was a utility developed to generate matrices of different dimensions and data-types. Furthermore, \textit{matrixgen} compiles the reference matrix \textbf{Ref} used when verifying the result returned from DirectCompute. \\
\textit{matrixgen} is written in \texttt{C++} and utilizes \texttt{C++~AMP} to generate and multiply matrices \textbf{A} \& \textbf{B} into product matrix \textbf{Ref}. In order to achieve random values in a \texttt{C++~AMP}-kernel the solution includes the random number generator-library \texttt{C++~AMP~RNG}.

\subsubsection*{experiment}
The second, \textit{experiment}, uses DirectCompute from the Microsoft~DirectX-framwork to compile the product matrix \textbf{C} from the matrices generated by \textit{matrixgen}. \textit{experiment} is written in \texttt{C++} with its respective DirectCompute-kernels written in \texttt{HLSL}. As \textit{experiment} is developed using the Windows~8 SDK, Windows~8 is required to run the application. \\
\\
In order to ease the experimental process, a script was developed to run both \textit{matrixgen} and \textit{experiment} consecutively a number of times with the desired configurations. This script, written in \texttt{Python}, then collect and present the results. \\
\\
The source code used during this study is freely available on \textit{GitHub}\footnote{Link here!}, along with a guide on how to compile and run the solution.

\subsubsection*{Equipment}
The experiment has been performed on a system with the following specifications:
\begin{description*}
	\item[CPU] \hfill \\
		Intel Q9550 Quad Core 2.83GHz
	\item[GPU] \hfill \\
		ATI Radeon HD 5800
	\item[OS] \hfill \\
		Windows 8.0
\end{description*}

% CONTRIBUTION - RESULTS
\subsection{Results}
\label{sec:contribution:results}

% Contribution
%	Vilka empiriska resultat erhölls vid experimentdesignens implementation?
% Conclusion
% ...for each segment brought up under Contribution.

% CONCLUSION
\section{Conclusion}
\label{sec:conclusion}
% Summary
%	Vilken bakgrund, utmaning, ansats och resultat omfattar studien?
% Discussion
%	Vilka relaterade studier förhåller sig materialet till?
%	Vilka tvetydliga aspekter av studien kräver diskussion?
% Answer and Oppertunity
%	Vilka fortsatta studier föreslår rapportens författare?

% CONCLUSION - FUTURE WORK
\subsection{Future~Work}
\label{sec:conclusion:futurework}

% ACKNOWLEDGMENTS
\section*{Acknowledgments} % The \section*{} command stops section numbering
\addcontentsline{toc}{section}{\hspace*{-\tocsep}Acknowledgments} % Adds this section to the table of contents with negative horizontal space equal to the indent for the numbered sections
\label{sec:acknowledgments}
I wish to express my gratitude to my fellow students, colleagues and close friends Bob, Benny and no-one for their work, co-operation and positive attitude throughout the execution of this experiment.

\bibliographystyle{IEEEtranS}
\bibliography{doc}
\end{document}

% To-be-inserted references and/or footnotes:
% Microsoft Warp: http://msdn.microsoft.com/en-us/library/windows/desktop/gg615082%28v=vs.85%29.aspx