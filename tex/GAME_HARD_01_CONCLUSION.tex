% GAME_HARD_01_CONCLUSION.tex

\section{CONCLUSION}
\label{sec:conclusion}
Based on the results presented in section~\nameref{sec:contribution:results}, using WARP to emulate the kernels presented in this study has magnitudes greater performance than if one were to apply the \textit{DirectX}~\textit{reference~device~driver} in the same manner.
Hence, if one were to compare the execution of WARP and the \textit{reference~device~driver} side-by-side, and assume the same area of application, WARP is superior in terms of execution time - assuming the same preconditions as those presented in this material.\\
However, keeping in mind the major performance improvements offered by Microsoft~WARP, it is important to consider that the two may be appropriate for different purposes.
The \textit{DirectX}~\textit{reference~device~driver} is primarily proposed by Microsoft as a debugging-/pre-silicon-development tool, whereas WARP is intended for use in a broader sense (see Microsoft's reference on WARP~\citeweb[]{warp}) - such as to render graphics for casual games - in addition to debugging and error-profiling purposes.
This calls for further inquiry into what the flaws of using Microsoft~WARP may be.\\
\\
In conclusion; this study proposes, pursuant to the established performance of Microsoft~WARP, that WARP-like technologies are feasable for extended use in applications - for purposes other than debugging and profiling.

% CONCLUSION - FUTURE WORK
\subsection{Future~Work}
\label{sec:conclusion:futurework}
For the sake of brevity, the author suggests complementary elaboration into double precision calculations in DirectCompute-kernels, with the intent of examining whether or not the Subjects detailed in section~\nameref{sec:contribution:subjectsofstudy} may have differentiating effects on computational precision.\\
\\
Additionally, the use of more complex kernels in coagency with the WARP-driver should be examined in order to study the effects of more demanding emulation on the CPU.
