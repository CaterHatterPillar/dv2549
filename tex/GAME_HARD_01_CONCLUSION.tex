% GAME_HARD_01_CONCLUSION.tex

\section{CONCLUSION}
\label{sec:conclusion}
Based on the findings of the experiment described in this paper, using WARP to simulate the presented kernels has magnitudes greater performance than if one were to apply the reference~device~driver in the same manner.
Hence, if one were to compare the execution of WARP and the reference~device~driver side-by-side, and assume the same area of application, WARP is superior in terms of execution time.

However, keeping in mind the major performance improvements offered by WARP, it is important to consider that the two may be appropriate for different purposes.
The reference~device~driver is primarily proposed by Microsoft as an accurate debugging-/pre-silicon-development tool (e.g., for the purposes of driver verification).
On the contrary, WARP is intended for use in a broader sense, such as to render graphics for casual games, in addition to error-profiling purposes~\citeweb[]{warp}.
Accordingly, the WARP device is included in the Direct3D~11~runtime for outside-development use.
This indicates potential for extended use above and beyond that of debugging- and profiling purposes, since the WARP device may be utilized where the reference~device~driver would otherwise be orders of magnitude too slow.
Such an example would be the isolation of hardware or graphics driver errors in high-performance graphics applications - often encountered by video game developers.

In conclusion; this study proposes, pursuant to the established performance of WARP, that WARP-like technologies are feasible for use where the reference~device~driver is too slow, such as verification of advanced computer graphics, or when hardware is not available, such as the acceleration of graphics in virtual platforms.

% CONCLUSION - FUTURE WORK
\subsection{Future~Work}
\label{sec:conclusion:futurework}
The performance improvements of WARP, in comparison to the reference~device~driver, calls for further inquiry into what the flaws of using WARP may be.
Microsoft lists, amongst other limitations latency-oriented CPUs may have over throughput oriented GPU architectures, memory bandwidth as a potential bottleneck for WARP-type acceleration methods~\citeweb[]{warp}.
In accordance to the suspicions raised by Microsoft, it may be beneficial to investigate the performance of WARP when such bottlenecks are stressed.
Furthermore, the author suggests complementary elaboration into double floating point precision calculations, with the intent of examining whether or not the Subjects detailed in section~\nameref{sec:contribution:subjectsofstudy} may have differentiating effects on computational precision.

Additionally, the use of more complex kernels in coagency with the WARP-driver should be examined in order to study the effects of more demanding simulation on CPUs.
